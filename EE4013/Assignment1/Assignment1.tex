\documentclass[journal,12pt,twocolumn]{IEEEtran}

\usepackage{setspace}
\usepackage{gensymb}
\singlespacing
\usepackage[cmex10]{amsmath}

\usepackage{amsthm}

\usepackage{mathrsfs}
\usepackage{txfonts}
\usepackage{stfloats}
\usepackage{bm}
\usepackage{cite}
\usepackage{cases}
\usepackage{subfig}

\usepackage{longtable}
\usepackage{multirow}

\usepackage{enumitem}
\usepackage{mathtools}
\usepackage{steinmetz}
\usepackage{tikz}
\usepackage{circuitikz}
\usepackage{verbatim}
\usepackage{tfrupee}
\usepackage[breaklinks=true]{hyperref}
\usepackage{graphicx}
\usepackage{tkz-euclide}

\usetikzlibrary{calc,math}
\usepackage{listings}
    \usepackage{color}                                            %%
    \usepackage{array}                                            %%
    \usepackage{longtable}                                        %%
    \usepackage{calc}                                             %%
    \usepackage{multirow}                                         %%
    \usepackage{hhline}                                           %%
    \usepackage{ifthen}                                           %%
    \usepackage{lscape}     
\usepackage{multicol}
\usepackage{chngcntr}

\DeclareMathOperator*{\Res}{Res}

\renewcommand\thesection{\arabic{section}}
\renewcommand\thesubsection{\thesection.\arabic{subsection}}
\renewcommand\thesubsubsection{\thesubsection.\arabic{subsubsection}}

\renewcommand\thesectiondis{\arabic{section}}
\renewcommand\thesubsectiondis{\thesectiondis.\arabic{subsection}}
\renewcommand\thesubsubsectiondis{\thesubsectiondis.\arabic{subsubsection}}


\hyphenation{op-tical net-works semi-conduc-tor}
\def\inputGnumericTable{}                                 %%

\lstset{
%language=C,
frame=single, 
breaklines=true,
columns=fullflexible
}
\begin{document}


\newtheorem{theorem}{Theorem}[section]
\newtheorem{problem}{Problem}
\newtheorem{proposition}{Proposition}[section]
\newtheorem{lemma}{Lemma}[section]
\newtheorem{corollary}[theorem]{Corollary}
\newtheorem{example}{Example}[section]
\newtheorem{definition}[problem]{Definition}

\newcommand{\BEQA}{\begin{eqnarray}}
\newcommand{\EEQA}{\end{eqnarray}}
\newcommand{\define}{\stackrel{\triangle}{=}}
\bibliographystyle{IEEEtran}
\raggedbottom
\setlength{\parindent}{0pt}
\providecommand{\mbf}{\mathbf}
\providecommand{\pr}[1]{\ensuremath{\Pr\left(#1\right)}}
\providecommand{\qfunc}[1]{\ensuremath{Q\left(#1\right)}}
\providecommand{\sbrak}[1]{\ensuremath{{}\left[#1\right]}}
\providecommand{\lsbrak}[1]{\ensuremath{{}\left[#1\right.}}
\providecommand{\rsbrak}[1]{\ensuremath{{}\left.#1\right]}}
\providecommand{\brak}[1]{\ensuremath{\left(#1\right)}}
\providecommand{\lbrak}[1]{\ensuremath{\left(#1\right.}}
\providecommand{\rbrak}[1]{\ensuremath{\left.#1\right)}}
\providecommand{\cbrak}[1]{\ensuremath{\left\{#1\right\}}}
\providecommand{\lcbrak}[1]{\ensuremath{\left\{#1\right.}}
\providecommand{\rcbrak}[1]{\ensuremath{\left.#1\right\}}}
\theoremstyle{remark}
\newtheorem{rem}{Remark}
\newcommand{\sgn}{\mathop{\mathrm{sgn}}}
% \providecommand{\abs}[1]{\left\vert#1\right\vert}
% \providecommand{\res}[1]{\Res\displaylimits_{#1}} 
% \providecommand{\norm}[1]{\left\lVert#1\right\rVert}
% %\providecommand{\norm}[1]{\lVert#1\rVert}
% \providecommand{\mtx}[1]{\mathbf{#1}}
% \providecommand{\mean}[1]{E\left[ #1 \right]}
\providecommand{\fourier}{\overset{\mathcal{F}}{ \rightleftharpoons}}
%\providecommand{\hilbert}{\overset{\mathcal{H}}{ \rightleftharpoons}}
\providecommand{\system}{\overset{\mathcal{H}}{ \longleftrightarrow}}
	%\newcommand{\solution}[2]{\textbf{Solution:}{#1}}
\newcommand{\solution}{\noindent \textbf{Solution: }}
\newcommand{\cosec}{\,\text{cosec}\,}
\providecommand{\dec}[2]{\ensuremath{\overset{#1}{\underset{#2}{\gtrless}}}}
\newcommand{\myvec}[1]{\ensuremath{\begin{pmatrix}#1\end{pmatrix}}}
\newcommand{\mydet}[1]{\ensuremath{\begin{vmatrix}#1\end{vmatrix}}}
\numberwithin{equation}{subsection}
\makeatletter
\@addtoreset{figure}{problem}
\makeatother
\let\StandardTheFigure\thefigure
\let\vec\mathbf
\renewcommand{\thefigure}{\theproblem}
\def\putbox#1#2#3{\makebox[0in][l]{\makebox[#1][l]{}\raisebox{\baselineskip}[0in][0in]{\raisebox{#2}[0in][0in]{#3}}}}
     \def\rightbox#1{\makebox[0in][r]{#1}}
     \def\centbox#1{\makebox[0in]{#1}}
     \def\topbox#1{\raisebox{-\baselineskip}[0in][0in]{#1}}
     \def\midbox#1{\raisebox{-0.5\baselineskip}[0in][0in]{#1}}
\vspace{3cm}
\title{Assignment 1}
\author{Neil Kamal Dhami - EE18BTECH11031}
\maketitle
\newpage
\bigskip
\renewcommand{\thefigure}{\theenumi}
\renewcommand{\thetable}{\theenumi}

Download all latex-tikz codes from 
\lstset{language=XML, basicstyle=\ttfamily}
\begin{lstlisting}
https://github.com/neildhami18/IITH_Academics/blob/main/EE4013/Assignment1/Assignment1.tex
\end{lstlisting}
\section{Problem}
(Q 18) Consider the following C program.

\lstset{language=C,
    basicstyle=\ttfamily,
    keywordstyle=\bfseries,
    showstringspaces=false,
    morekeywords={include, printf}
}

\begin{lstlisting}
#include <stdio.h>
int jumble(int x, int y){
    y = 2*x + y;
    return y;
}
int main(){
    int x=2, y=5;
    y = jumble(y,x);
    x = jumble(y,x);
    printf("%d \n", x);
    return 0;
}
\end{lstlisting}
The value printed by the program is?

\section{Solution}
\textbf{Answer} : 26
\newline

\textbf{Explanation:}
\newline
This is a very simple function call problem. The only tricky part involved is that the global variables \textbf{x} and \textbf{y} are called as parameters $y$ and $x$ respectively during function call. The function calls in the main loop return the values for \textbf{x} and \textbf{y} as:
\begin{lstlisting}
    y = jumble(y,x) = 2*5 + 2 = 12
    x = jumble(y,x) = 2*12 + 2 = 26
\end{lstlisting}

\section{Concept}
The function \textbf{jumble} performs an operation commonly known as "\textbf{saxpy}" (\textbf{S}ingle-precision real \textbf{A}lpha \textbf{X} \textbf{P}lus \textbf{Y})

\begin{definition}[Saxpy]
If x, y $\in \mathbb{R}^{n}$ and a $\in \mathbb{R}$, then this operation overwrites y with y + a*x. 
\end{definition}

This function can also be presented as a dot product of a coefficient vector with the input vector:
\begin{align}
    y = \begin{pmatrix} a & 1 \end{pmatrix} 
    \begin{pmatrix} x \\ y \end{pmatrix}
\end{align}

where a = 2 in our case.

\section{Application}

Let us utilise data structures to save two vectors and develop a function to generate their dot product.\\
We employ linked lists to store our data since they are dynamic Data Structures and allow utilisation of simple insertion/deletion functions.\\
Linked List Data Structure:
\lstset{language=C,
    basicstyle=\ttfamily,
    keywordstyle=\bfseries,
    showstringspaces=false,
    morekeywords={include, printf}
}

\begin{lstlisting}
    struct node {
       int data;
       int key;
       struct node *next;
    };
\end{lstlisting}

Given repository contains full code.
\lstset{language=XML, basicstyle=\ttfamily}
\begin{lstlisting}
https://github.com/neildhami18/IITH_Academics/blob/main/EE4013/Assignment1/codes/dot_product.c
\end{lstlisting}
Glimpse of the function performing the dot operation of two linked lists:
\lstset{language=C,
    basicstyle=\ttfamily,
    keywordstyle=\bfseries,
    showstringspaces=false,
    morekeywords={include, printf}
}

\begin{lstlisting}
int dot(struct node* List1, struct node* List2)
{
    int product = 0;
    struct node *current_a = List1;
    struct node *current_b = List2;

    while(current_a != 0 && current_b!=0)
    {
        if(current_a->key == current_b->key)
        {
            product = product + current_a->data * current_b->data;
            current_a=current_a->next;
            current_b=current_b->next;
        }
        else if(current_a->key < current_b->key)
        {
            current_a=current_a->next;
        }
        else
        {
            current_b=current_b->next;
        }
    }
    return product;
}

\end{lstlisting}

\section{File Handling}

I have also added a code segment(extention) in order to facilitate reading of vector data from text files into the linked lists.\\
File Handling Code Snippet:
\begin{lstlisting}
    f = fopen("vector.txt", "r");
       int i=1;
       while(fgets(line, sizeof(line), f))
       {
           LIST *node = malloc(sizeof(LIST));
           node->string = strdup(line);
           int data = atoi(node->string);
           List = insertFirst(List,i,data);
           i++;
       }
       fclose(f);

\end{lstlisting}


\end{document}

